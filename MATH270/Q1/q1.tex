\documentclass{article}

% Language setting
% Replace `english' with e.g. `spanish' to change the document language
\usepackage[english]{babel}

% Set page size and margins
% Replace `letterpaper' with `a4paper' for UK/EU standard size
\usepackage[letterpaper,top=2cm,bottom=2cm,left=3cm,right=3cm,marginparwidth=1.75cm]{geometry}

% Useful packages
\usepackage{amsmath}
\usepackage{amssymb}
\usepackage{graphicx}
\usepackage[colorlinks=true, allcolors=blue]{hyperref}

\title{MATH 270 Unit 1}
\author{Miles Kent}

\begin{document}
\maketitle

\section{Test Necessities}
\begin{enumerate}
    \item Be able to determine a particular entry in the product of two matrices; the entry in a particular row and column of the matrix product.

        Let $A_k$ be the $k^{th}$ row of the matrix $A$

        Let $b_k$ be the $k^{th}$ column of the matrix $B$

        $(AB)_{ij} = \vec{A}_i \cdot \vec{b}_j$


    \item Be able to determine if two matrices can be multiplied together and if so determine the dimensions of the product

        Let $A$ equal the 3x4 matrix
        $\begin{bmatrix}
        a & b & c & d\\
        e & f & g & h\\
        i & j & k & l
        \end{bmatrix}$

        Let $B$ equal the 4x2 matrix
        $\begin{bmatrix}
        A & B\\
        C & D\\
        E & F\\
        G & H
        \end{bmatrix}$

        The row and columns of a matrix are denoted as rows x columns.
        For example, A is a 3x4 matrix. Therefore, it has 3 rows and 4 columns.
        The order of matrix multiplication matters.
        $A$ and $B$, for example, can only be multiplied $AB$, not $BA$.
        Look at the dimensions of the matrices. The inner dimensions need to match for multiplication
        to be possible. The outer dimensions specify the new matrix.
        $AB$ is 3x4 4x2. The inner dimensions match 3x$\boldsymbol{4\ 4}$x2.
        The outer dimensions $\boldsymbol{3}$x4 4x$\boldsymbol{2}$ indicate that AB will be 3x2.

    \item Be able to determine if a transformation between vector spaces is a linear transformation.

        A transformation $T$ is a linear transformation if

        1. $T(\vec{u}_1 + \vec{u}_2) = T(\vec{u}_1) + T(\vec{u}_2)$

        2. $T(c\vec{u}_1) = cT(\vec{u}_1)$

    \item Be able to write down the transformation matrix of some given linear transformation.

        Example: $ T: R^2 \longrightarrow R^2 $. Rotation CC by $\theta$
        Take the basis vectors for the given vector space and transform them by the linear map.
        The basis vectors for $R$ are $\vec{e}_1, \vec{e}_2$.\\

        $T(\vec{e}_1) =
        T(\begin{bmatrix}
        1\\
        0
        \end{bmatrix})
        =
        \begin{bmatrix}
        cos(\theta)\\
        sin(\theta)
        \end{bmatrix}$

        $T(\vec{e}_2) =
        T(\begin{bmatrix}
        0\\
        1
        \end{bmatrix})
        =
        \begin{bmatrix}
        -sin(\theta)\\
        cos(\theta)
        \end{bmatrix}$

        $\therefore
        \begin{bmatrix}
            T
        \end{bmatrix}
        =
        \begin{bmatrix}
        cos(\theta) & -sin(\theta)\\
        sin(\theta) & cos(\theta)
        \end{bmatrix}
        $


    \item Be able to find the length of a vector.

    $|\vec{A}| = \sqrt{A_x^2 + A_y^2}$

    \item Be able to find a unit vector (vector of length 1) in the same direction as a given vector.

    $\vec{u} = \frac{\vec{A}}{|\vec{A}|}$
    \item Be able to find the angle between two vectors.

    $cos(\theta) = \frac{\vec{A} \cdot \vec{B}}{|\vec{A}||\vec{B}|}$
    \item Be able to determine if two vectors are perpendicular.

    When $\vec{A} \cdot \vec{B} = 0$\\
    $cos(90^\circ) = \frac{\vec{A} \cdot \vec{B}}{|\vec{A}||\vec{B}|} = 0$
\end{enumerate}


\end{document}
